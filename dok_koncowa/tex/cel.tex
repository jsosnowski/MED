\section{Definicja zadania}

Tematem projektu jest implementacja i porównanie algorytmów CLOSET \cite{closetArt} i CHARM \cite{charmArt}. Wykorzystuje się je do odkrywania domkniętych zbiorów częstych w zbiorach transakcji.
Są to konkurencyjne podejścia. Pierwsze powstało w Kanadzie, a jego autorzy to m.in. Jian Pei i Jiawei Han. Drugi algorytm został opublikowany m.in. w artykule z 2002 roku na konferencji SIAM. Jego autorami są Mohammed Zaki i Ching-Jui Hsiao. W niniejszej pracy korzystamy również z młodszej pracy o takim samym tytule opublikowanej w roku 2005 \cite{charmArt}. \\

Dokładnym celem projektu była implementacja obu algorytmów z wykorzystaniem języka Java w wersji 8. Algorytmy, po sprawdzeniu poprawności ich działania, należało je przetestować i porównać pod kątem wydajności.

Naszym osobistym wkładem była implementacja od podstaw obu algorytmów, bazując jedynie na wiedzy zawartej w artykułach ich autorów, a następnie opracowanie testów i porównanie obu metod szukania domkniętych zbiorów częstych.