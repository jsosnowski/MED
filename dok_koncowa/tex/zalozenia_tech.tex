\section{Realizacja projektu}

\subsection{Implementacja}

Implementacje obu algorytmów to oddzielne aplikacje konsolowe napisane w języku Java 8. Każdy z członków zespołu zajął się implementacją jednego z algorytmów, co pozwoliło na zrównoleglenie pracy. Do budowania aplikacji wykorzystano \emph{Gradle}.

Implementując oba algorytmy nie korzystano z żadnych dodatkowych bibliotek wykonujących istotną część zadania -- wykorzystano natomiast szereg bibliotek pomocniczych:

\begin{itemize}
	\item \emph{Open CSV} -- do obsługi wczytywania danych z plików \emph{CSV}
	\item \emph{Logback} -- obsługa logowania
	\item \emph{Google Guava}, \emph{Apache Commons} -- rozmaite narzędzia ,,utility''
	\item \emph{JUnit}, \emph{AssertJ} -- testy jednostkowe
\end{itemize}

\subsection{Budowanie aplikacji}

Obie aplikacje można wygodnie zbudować za pomocą systemu \emph{Gradle}. Komendy:
\begin{lstlisting}
./gradlew clean
./gradlew build
\end{lstlisting}

W wyniku wywołania komend w katalogach $build/lib/$ pojawią się wykonywalne pliki Java z rozszerzeniem $*.jar$.

\subsection{Uruchamianie}


Oba stworzone w ramach projektu programy można uruchomić z poziomu konsoli.

Dane wejściowe, na których testowane były algorytmy pochodziły z repozytorium UCI \cite{UCI} -- zbiór danych to ,,\emph{Mushroom Database}''. Zbiór ten zawiera dane o 8124 grzybach, z których każdy opisany jest 22 atrybutami.

\begin{itemize}
  \item Obie implementacje będą oddzielnymi aplikacjami konsolowymi.
  \item Językiem programowania będzie Java 8.
  \item System budowania opierać się będzie o technologię Gradle.
  \item Dane wejściowe będą stanowić zbiory danych z repozytorium UCI \cite{UCI}.
  \item Dane wyjściowe będą wyprowadzane na standardowe wyjście lub do pliku.
  \item Wszelkie użycia dodatkowych bibliotek nie będą związane z główną funkcjonalnością programów, a jedynie z funkcjami pomocniczymi (np. z logowaniem, czy z testowaniem aplikacji).
\end{itemize}
