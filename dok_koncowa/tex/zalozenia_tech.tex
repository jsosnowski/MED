\section{Realizacja projektu}

\subsection{Implementacja}

Implementacje obu algorytmów to oddzielne aplikacje konsolowe napisane w języku Java 8. Każdy z członków zespołu zajął się implementacją jednego z algorytmów, co pozwoliło na zrównoleglenie pracy. Do budowania aplikacji wykorzystano \emph{Gradle}.

Implementując oba algorytmy nie korzystano z żadnych dodatkowych bibliotek wykonujących istotną część zadania -- wykorzystano natomiast szereg bibliotek pomocniczych:

\begin{itemize}
	\item \emph{Open CSV} -- do obsługi wczytywania danych z plików \emph{CSV}
	\item \emph{Logback} -- obsługa logowania
	\item \emph{Google Guava}, \emph{Apache Commons} -- rozmaite narzędzia ,,utility''
	\item \emph{JUnit}, \emph{AssertJ} -- testy jednostkowe
\end{itemize}

\subsection{Budowanie aplikacji}

Obie aplikacje można wygodnie zbudować za pomocą systemu \emph{Gradle}. Komendy:
\begin{lstlisting}
./gradlew clean
./gradlew build
\end{lstlisting}

W wyniku wywołania komend w katalogach $build/libs/$ pojawią się wykonywalne pliki Java z rozszerzeniem $*.jar$.

\subsection{Uruchamianie}

Obie aplikacje uruchamia się bardzo podobnie. Algorytm CHARM:
\begin{lstlisting}
java -jar build/libs/med-charm-1.0.jar <opcje>
\end{lstlisting}
gdzie opcje:
\begin{itemize}
	\item <ścieżka do pliku z danymi CSV> -- Plik wejściowy z danymi CSV musi zawierać nagłówek opisujący nazwy atrybutów w transakcjach; przykładowy plik znajduje się w katalogu $src/test/resources/$
	\item <minimalne wsparcie relatywne> -- Minimalne wsparcie, np. 0.5
	\item $[$ścieżka do pliku output$]$ -- Plik wyjściowy jest nieobowiązkowy i służy głównie do testów; w przypadku jego podania aplikacja tworzy plik, w którym znajduje się tylko jedna liczba -- czas wykonania algorytmu w nanosekundach
\end{itemize}

Algorytm CLOSET:
\begin{lstlisting}
java -jar build/libs/med-closet-1.0.jar <opcje>
\end{lstlisting}
gdzie opcje:
\begin{itemize}
	\item <ścieżka do pliku z danymi CSV> -- Plik wejściowy z danymi CSV
	\item <minimalne wsparcie relatywne> -- Minimalne wsparcie, np. 0.5
\end{itemize}

Oba programy zwracają wyniki działania także na standardowe wyjście.