\documentclass[12pt,oneside,a4paper]{article}
%\documentclass{llncs}
\usepackage[utf8x]{inputenc}
\usepackage[T1]{fontenc}
\usepackage[english,polish]{babel}
\usepackage{polski}
\usepackage{float}
\usepackage{hyperref} %spis treści niech będzie łączami
\usepackage{listings}
\hypersetup{
	plainpages=false,
    colorlinks,
    citecolor=black,
    filecolor=black,
    linkcolor=black,
    urlcolor=black
}
\usepackage{graphicx}
\usepackage{pbox}
\usepackage{tabto}
\usepackage{algorithm}
\usepackage{algpseudocode}
% ustawienia środowiska algorytm
\floatname{algorithm}{Algorytm}
\renewcommand{\algorithmicfunction}{\textbf{funkcja}}
% koniec ustawień

% blokada dzielenia wyrazów (dla kopiowania do Worda)
% \hyphenpenalty 10000
% \exhyphenpenalty 10000
% koniec ustawień blokady

\begin{document}
\author{
  Jacek Sosnowski
  \and
  Michał Cybulski
}

\title{Implementacja i porównanie algorytmów CHARM i CLOSET}
\pagenumbering{alph}
\maketitle
%\input{tex/strona_tytulowa.tex}
\clearpage
\tableofcontents
\clearpage
\pagenumbering{arabic}

\section{Definicja zadania}

Tematem projektu jest implementacja i porównanie algorytmów CLOSET \cite{closetArt} i CHARM \cite{charmArt}. Wykorzystuje się je do odkrywania domkniętych zbiorów częstych w zbiorach transakcji.
Są to konkurencyjne podejścia. Pierwsze powstało w Kanadzie, a jego autorzy to m.in. Jian Pei i Jiawei Han. Drugi algorytm został opublikowany m.in. w artykule z 2002 roku na konferencji SIAM. Jego autorami są Mohammed Zaki i Ching-Jui Hsiao. W niniejszej pracy korzystamy również z młodszej pracy o takim samym tytule opublikowanej w roku 2005 \cite{charmArt}. \\

Dokładnym celem projektu była implementacja obu algorytmów z wykorzystaniem języka Java w wersji 8. Algorytmy, po sprawdzeniu poprawności ich działania, należało je przetestować i porównać pod kątem wydajności.

Naszym osobistym wkładem była implementacja od podstaw obu algorytmów, bazując jedynie na wiedzy zawartej w artykułach ich autorów, a następnie opracowanie testów i porównanie obu metod szukania domkniętych zbiorów częstych.

\section{Algorytm Charm}

Algorytm Charm opisany w artykule \cite{charmArt} służy do wyszukiwania wszystkich domkniętych zbiorów częstych w zbiorze danych transakcyjnych. Sposób jego działania pozwala na bardzo tanie pozbywanie się zbiorów redundantnych (i takich, które będą generowały redundantne zbiory), co w praktyce w bardzo dużym stopniu ogranicza złożoność obliczeniową potrzebną do wygenerowania zbiorów częstych. Algorytm został przedstawiony w pseudo-kodzie \ref{charmAlgorithm}.

\begin{algorithm}
\caption{Algorytm Charm}
\label{charmAlgorithm}
\renewcommand{\algorithmicrequire}{\textbf{Wejście:}}
\renewcommand{\algorithmicensure}{\textbf{Wyjście:}}
\begin{algorithmic}
	\Require Transakcyjna baza danych TDB i próg minimalnego wsparcia min\_sup.
	\Ensure  Kompletny zbiór częstych zbiorów domkniętych.
	\Function{charm}{$TDB \subseteq \mathcal{I} \times \mathcal{T}$, $min\_sup$}
		\State $nodes \gets \{I_{j} \times t(I_{j}) : I_{j} \in \mathcal{I} \wedge |t(I_{j})| \ge min\_sup \} $
		\State $FCI \gets \emptyset$ \Comment{$FCI$ to zbiór częstych zbiorów zamkniętych}
		\State $\Call{charmExtend}{nodes, FCI}$
		\State \Return $FCI$
	\EndFunction
	\Function{charmExtend}{$nodes$, $FCI$}
		\ForAll {$X_{i} \times t(X_{i}) \in nodes$}
			\State $newN \gets \emptyset$
			\State $X = X_{i}$
			\ForAll {$ X_{j} \times t(X_{j}) \in nodes : f(j) > f(i) $} \Comment{$f(a)$ - funkcja ustalająca kolejność (np. wg kolejności leksykalnej lub wsparcia)}
				\State $X = X \cup X_{j}$
				\State $Y = t(X_{i}) \cap t(X_{j})$
				\State $\Call{charmProperty}{nodes, newN}$
			\EndFor
			\If {$newN \neq \emptyset $}
				\State $\Call{charmExtend}{newN, FCI}$
			\EndIf
		\EndFor
	\EndFunction
	\Function{charmProperty}{$nodes, newN$}
		\If{$|Y| > min\_sup$}
			\If{$t(X_{i}) = t(X_{j})$}
				\State Remove $X_{j}$ from $nodes$
				\State Replace all $X_{i}$ with $X$
			\ElsIf{$t(X_{i}) \subset t(X_{j})$}
				\State Replace all $X_{i}$ with $X$
			\ElsIf{$t(X_{i}) \supset t(X_{j})$}
				\State Remove $X_{j}$ from $nodes$
				\State Add $X \times Y$ to $newN$
			\ElsIf{$t(X_{i}) \neq t(X_{j})$}
				\State Add $X \times Y$ to $newN$
			\EndIf
		\EndIf
	\EndFunction
\end{algorithmic}
\end{algorithm}

Cechą wyróżniającą algorytm Charm jest to, że algorytm ten działa praktycznie jednocześnie zarówno na przestrzeni elementów jak i na przestrzeni identyfikatorów transakcji, podczas gdy większość algorytmów porusza się wyłącznie po przestrzeni elementów. Ta cecha pozwala na wykorzystanie bardzo efektywnej metody badania czy rozpatrywany zbiór jest zbiorem domkniętym i ignorowanie go w przeciwnym wypadku. Analizowane zbiory nie są więc eliminowane tylko na podstawie małej częstości ich występowania, ale także na podstawie braku ich domknięcia.

Algorytm Charm nie korzysta w swojej implementacji z drzew, jak typowe algorytmy generujące zbiory częste. Autorzy sugerują natomiast, żeby dane trzymać w postaci ,,pionowej'' (ang. ,,vertical data format''), czyli niejako przyporządkowując każdemu elementowi zbiór identyfikatorów transakcji, w których on występuje. Dzięki temu operacje sumy oraz przecięć zbiorów oraz wykrywanie czy zbiór transakcji jednego elementu zawiera się w zbiorze drugiego elementu (lub vice versa) są bardzo tanie (praktycznie darmowe) -- zwłaszcza w porównaniu z sytuacją, gdy dane trzymane są standardowo, czyli zbiory elementów przyporządkowywane są poszczególnym transakcjom, a takie porównanie oznacza konieczność iteracji po wszystkich transakcjach.

\section{Algorytm Closet}

Oto opis algorytmu Closet:



\section{Realizacja projektu}

\subsection{Implementacja}

Implementacje obu algorytmów to oddzielne aplikacje konsolowe napisane w języku Java 8. Każdy z członków zespołu zajął się implementacją jednego z algorytmów, co pozwoliło na zrównoleglenie pracy. Do budowania aplikacji wykorzystano \emph{Gradle}.

Implementując oba algorytmy nie korzystano z żadnych dodatkowych bibliotek wykonujących istotną część zadania -- wykorzystano natomiast szereg bibliotek pomocniczych:

\begin{itemize}
	\item \emph{Open CSV} -- do obsługi wczytywania danych z plików \emph{CSV}
	\item \emph{Logback} -- obsługa logowania
	\item \emph{Google Guava}, \emph{Apache Commons} -- rozmaite narzędzia ,,utility''
	\item \emph{JUnit}, \emph{AssertJ} -- testy jednostkowe
\end{itemize}

\subsection{Budowanie aplikacji}

Obie aplikacje można wygodnie zbudować za pomocą systemu \emph{Gradle}. Komendy:
\begin{lstlisting}
./gradlew clean
./gradlew build
\end{lstlisting}

W wyniku wywołania komend w katalogach $build/libs/$ pojawią się wykonywalne pliki Java z rozszerzeniem $*.jar$.

\subsection{Uruchamianie}

Obie aplikacje uruchamia się bardzo podobnie. Algorytm CHARM:
\begin{lstlisting}
java -jar build/libs/med-charm-1.0.jar <opcje>
\end{lstlisting}
gdzie opcje:
\begin{itemize}
	\item <ścieżka do pliku z danymi CSV> -- Plik wejściowy z danymi CSV musi zawierać nagłówek opisujący nazwy atrybutów w transakcjach; przykładowy plik znajduje się w katalogu $src/test/resources/$
	\item <minimalne wsparcie relatywne> -- Minimalne wsparcie, np. 0.5
	\item $[$ścieżka do pliku output$]$ -- Plik wyjściowy jest nieobowiązkowy i służy głównie do testów; w przypadku jego podania aplikacja tworzy plik, w którym znajduje się tylko jedna liczba -- czas wykonania algorytmu w nanosekundach
\end{itemize}

Algorytm CLOSET:
\begin{lstlisting}
java -jar build/libs/med-closet-1.0.jar <opcje>
\end{lstlisting}
gdzie opcje:
\begin{itemize}
	\item <ścieżka do pliku z danymi CSV> -- Plik wejściowy z danymi CSV musi zawierać nagłówek opisujący nazwy atrybutów w transakcjach; przykładowy plik znajduje się w katalogu $src/test/resources/$
	\item <minimalne wsparcie relatywne> -- Minimalne wsparcie, np. 0.5
\end{itemize}

Oba programy zwracają wyniki działania także na standardowe wyjście.

\clearpage
\bibliographystyle{splncs}
\begin{thebibliography}{99}

\bibitem{charmArt} \emph{CHARM: An Efficient Algorithm for Closed Association Rule Mining}, Mohammed J. Zaki, Ching-Jui Hsiao, 2005

\bibitem{closetArt} \emph{CLOSET: An Efficient Algorithm for Mining Frequent Closed Itemsets}, Jian Pei, Jiawei Han, Runying Mao, 2000

\bibitem{UCI} \emph{UCI Machine Learning Repository}, http://archive.ics.uci.edu/ml/ [dostęp: 12.12.2015r.]

\end{thebibliography}

\end{document}
