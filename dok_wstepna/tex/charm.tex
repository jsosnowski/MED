\section{Algorytm Charm}

Tu będzie pięknie napisany opis algorytmu Charm w pseudokodzie.
Ale z racji tego, że jeszcze go nie ma, to wklejam inny algorytm ;) To tak tylko dla przykładu, żeby było się na czym wzorować.

\begin{algorithm}
\caption{Procedura grupowania Quality Threshold Clustering}
\label{AlgQTC}
\begin{algorithmic}
	\Function{QTClustering}{G, d} 
		\If{$\vert G\vert \leq 1$} 	
		\Return G \EndIf
		
		\ForAll{ $i \in G$ }
			 \State zbiór $A_{i} \gets \{ i \}$ \Comment{$A_{i}$ jest i-tym kandydatem}
			 \While {$A_{i} \neq G$}
			 	\State znajdź $j \in ( G - A_{i} )$ dla którego \Call{średnica}{$A_{i} \cup {j}$} jest minimalna
			 	
			 	\If{\Call{średnica}{$A_{i} \cup {j}$} < d}
			 		\State $A_{i} \gets A_{i} \cup \{j\}$
			 	\Else
			 		\State \textbf{break while}
			 	\EndIf
	
			 \EndWhile
		\EndFor
		
		\State zbiór $C \gets$ \Call{największy\_zbiór\_z}{ $ A_{1}, A_{2}, A_{3}, \ldots , A_{\vert G \vert} $ }
		
		\State \Return \{ $C$, \Call{QTClustering}{$G - C, d$} \}

	\EndFunction
\end{algorithmic}
\end{algorithm}



