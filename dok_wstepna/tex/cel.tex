\section{Cel pracy}

Celem projektu jest implementacja obu algorytmów z wykorzystaniem języka Java w wersji 8.
Kolejnym kluczowym etapem będzie sprawdzenie poprawności obu implementacji.
Pozytywne zakończenie tego etapu pozwoli na wykonanie szeregu testów wydajnościowych służących analizie porównawczej obu algorytmów.
Naszym priorytetem jest sprawdzenie dwóch implementacji wykonanych we współczesnym języku programowania i znalezienie odpowiedzi na pytanie: które z tych dwóch podejść jest wydajniejsze?

Jako, że oba algorytmy powinny zwracać te same wyniki (w przeciwnym wypadku zostały błędnie zaimplementowane), to porównując je skupimy się głównie na badaniu różnic wydajności obu podejść, a także na badaniu ich skalowalności.

Zamierzamy odnosić się również do wyników eksperymentów opisanych w pracach autorstwa J.Pei \cite{closetArt} w celu weryfikacji wniosków tam przedstawionych.

Naszym osobistym wkładem będzie implementacja od podstaw obu algorytmów, bazując jedynie na wiedzy zawartej w artykułach ich autorów, a następnie opracowanie testów i porównanie obu metod szukania domkniętych zbiorów częstych.
Poniżej przedstawiamy wstępne wprowadzenie teoretyczne, na którym oparte będą dalsze prace.