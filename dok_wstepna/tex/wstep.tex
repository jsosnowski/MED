\section{Wstęp}

Tematem projektu są algorytmy CLOSET \cite{closetArt} i CHARM \cite{charmArt}. Wykorzystuje się je do odkrywania domkniętych zbiorów częstych.
Są to konkurencyjne podejścia. Pierwsze powstało w Kanadzie, a jego autorzy to m.in. Jian Pei i Jiawei Han. Drugi algorytm został opublikowany m.in. w artykule z 2002 roku na konferencji SIAM. Jego autorami są Mohammed Zaki i Ching-Jui Hsiao. W niniejszej pracy korzystamy również z młodszej pracy o takim samym tytule opublikowanej w roku 2005 \cite{charmArt}.

Celem projektu jest implementacja obu algorytmów z wykorzystaniem języka Java w wersji 8.
Kolejnym kluczowym etapem będzie sprawdzenie poprawności obu implementacji.
Pozytywne zakończenie tego etapu pozwoli na wykonanie szeregu testów wydajnościowych służących analizie porównawczej obu algorytmów.
Naszym priorytetem jest sprawdzenie dwóch implementacji wykonanych we współczesnym języku programowania i znalezienie odpowiedzi na pytanie: które z tych dwóch podejść jest wydajniejsze?
Zamierzamy odnosić się również do wyników eksperymentów opisanych w pracach autorstwa J.Pei \cite{closetArt} w celu weryfikacji wniosków tam przedstawionych.
